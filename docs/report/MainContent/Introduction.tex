\section{Introduction}
Robotic locomotion has, until a few years ago, been focused on wheel-based movement. 
Although it is very stable and easy to implement, it lacks flexibility, the ability to move on uneven, unpredictable terrain and overcome obstacles such as stairs.
\par As a RoboCup team Bold Hearts member, which competes in the humanoid soccer league, our robots must walk, a recurring problem due to rule changes. 
As the objective of RoboCup is to achieve a realistic environment, competition rules change regularly. 
Changes in rules involve field changes, such as moving from flat ground to synthetic grass, enforcing that teams develop walking algorithms that can adapt to more variable environments. 
Rule changes also affect the robots, including their height, types of sensors and others. 
Changes in the robot's structure lead to the need to readapt the walking algorithms as they are dependent on these variables. 
These changes are time-consuming, and walking algorithms are a complex task requiring much effort from the team.
\subsection{Intro to the project}