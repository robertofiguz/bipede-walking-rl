
\section{Cartpole Outcomes}
The first task developed was the classic cartpole environment,
this was helpfull in understanding core concepts of reinforcement learning and neural networks, along with it,
cartpole was essential in testing and setting up the logging interface as well as testing different implementations of the learning algorithm 
\subsection*{Keras-rl}
Keras-rl is a community maintained high level implementation of keras agents for reinforcement learning, this was the first implementaiton tested. 



The implementation of keras-rl is very easy and diesn't require a deep understanding of reinforcement learning and the leanring structure.
\subsection*{Keras API}
The second implementaiton tested was using the plain keras API, while this provides more flexibility it also requires a much deeper understandng of how reinforcement learning works.
%%%%%%%%% INSERT DETAILS HERE %%%%%%%%%
The implementaiton using the Keras API was essential to develop a necessary knowledge for the project and to progress to the next stage.
\subsection*{}

While an implementaiton using Keras-rl would be simpler and even possibly ease the itteration process, this implementation provides less flexibility and given the target of the project and 
desire to develop a deeper understanding of reinforcement learning the implementation using the Keras API was choosen to implemnt the next phases.

\subsection*{Results}

%%%%%%%%% INSERT RESULTS DATA AND ANALYSIS HERE %%%%%%%%%%%%%%%%%%

\section{2D Environment Outcomes}

\section{3D Environment Outcomes}

\section{Reward Function}

