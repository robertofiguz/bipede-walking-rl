\chapter{Future Research}
Reinforcement learning is a very complex topic, specifically when applied to such a complex movement as robotic bipedal walking. There are many approaches to solving reinforcement learning problems. Although many areas of interest couldn't be covered, this can be explored in future research on Bipedal Robotic walking and reinforcement learning in general.

As was shown in this report, developing an optimal reward function can be complex, which is an unsolved problem in the reinforcement learning field. One promising topic suggested by the project supervisor was empowerment\cite{empowerment} by intrinsic motivations. Empowerment is a technique that aims to overcome the reward problem by equipping the robot with intrinsic rewards, using rewards such as curiosity and empowerment.

The developed project uses discrete actions although, there is a loss of information when using this method, for future research, it would be valuable to compare a policy gradient approach using continuous actions.

The initial design for this project was to use Mujoco as a physics engine. However, due to time constraints, the plan had to be changed to use Gazebo as the environment was already modelled and integrated directly with ROS2. Nonetheless, the use of Mujoco is still of interest to the team, and a comparison against Gazebo can be researched.

One of the problems when developing something as simple as a motion script for the robot is the transition to the real robot, while the script might work perfectly in the simulator, it will require adaptation to work on the real robot. This is a topic of interest for the future and understanding how, using the simulator to train most of the movement, the learning can be transferred to the real robot and how to tune the learned model when testing on the robot.

