\section{Development Structure}
Due to the complexity of the project, a development structure has been put in place, 
this includes multiple steps of increasing complexity and realism, 
the increasing complexity allows for detecting problems at earlier, simpler stages, making the transition and understanding the problems easier.
\subsection{Cartpole}
Cartpole is a classic exercise of reinforcement learning, it consists in balancing a pole in a cart moving on an horizontal plane by applying a force on the right or left side of the cart making it move in the oposite direction. 

The cartpole environment allowes for the implementation and testing of the reinforcement learning algorithm, different implementations and its comparison, at this stage it was also used to implement the logging and reproducibility interface.

\subsection{2D Walker}
At this stage the complexity of the environment increases as the environment starts to assimilate the target problem. Although, this stage eliminates some of the complexity such as using a more complex 3D environment and implementing the training with the robot controll interface.
This allows for the implementation of a neural network, learning algorithm that is capable of handling multiple simultaneous outputs as it is required to control all the joints of the robot and understanding how to efficiently calculate the best action.

To achieve this it is necessary to develop a custom 2D environment of a simplified humanoid in order to train a walking behaviour.
New challenges from this stage such as implementing a custom reward system, rendering and step functions are an important step in order to transition to 3D simulation.

\subsection{3D Walker}
The final stage of the project is the implementation on a 3D simulated robot, this is the combination of the previous stage with extre complexity, not only due to the inherited complexity of a higher diensional world but because this should be able to integrate with the real robot from the Bold Hearts team and therefore use its control interface.
Robot simulation is the main platform for developing software for robotics, it has many benefits, developing software and testing it directly on a real robot can be a very slow process and can even lead to breaking the robot.

3d simulation brings new challenges, such as a larger range of motion and more joints to controll, along with a more complex environment, requiring more processing power and more time to solve the problem. 
Along with this it requires a more complex reward system as a new dimention poses new problems.


% values for reward goes in experiment
% what goes in the reward system goes in design


%design
%implementation
%experimental results

%author and year - reference