\chapter{Project Evaluation}
Walking for bipedal robots using reinforcement learning was an ambitious project. Although walking could not be achieved, this project successfully developed a working 2D walking environment with a simple humanoid, implementing OpenAI Gym with ROS2 and developing documentation and preparation for future research on the topic.

The training for this project was mainly executed using Google colab+, as incompatibilities with the computer architecture (ARM64) made training locally slower in comparison. Although, Google colab constantly crashed due to timeouts and unknown problems, making it impossible to train for extended periods of time. 
On a retrospective, it would have been a better decision to train locally as even if the training was slower, it would be able to run for very long periods of time.

As already explored, the reward function should have been tested from the beginning exclusively with positive rewards, as this would have allowed for a more efficient experiment.



Although no training was executed, the last stage was a significant achievement as it is a useful tool and development not exclusive to this project but for the team. 